Consider the three nonzero vectors
\[
v_1 =
\begin{bmatrix}
1 \\
-2 \\
3
\end{bmatrix}, \quad
v_2 =
\begin{bmatrix}
2 \\
0 \\
1
\end{bmatrix}, \quad
v_3 =
\begin{bmatrix}
3 \\
-2 \\
1
\end{bmatrix}.
\]

\begin{enumerate}
    \item[(a)] Show that $v_1$ does not belong to the span of $v_2$ and $v_3$. (Hint: if $v_1 = a v_2 + b v_3$ for some scalars $a$ and $b$, express this as a system of 3 equations on $a$ and $b$ and show that these equations have no simultaneous solution.)
    \item[(b)] Similarly show $v_2$ does not belong to the span of $v_1$ and $v_3$, and that $v_3$ does not belong to the span of $v_1$ and $v_2$.
    \item[(c)] Using (a) and (b), apply Theorem 4.2.5 to conclude that the linear subspace $V = \text{span}(v_1, v_2, v_3)$ in $\mathbb{R}^3$ has dimension equal to 3 (so it coincides with $\mathbb{R}^3$, by Theorem 4.2.8).
\end{enumerate}

Solution:


\begin{enumerate}
    \item[(a)] Show that $v_1$ does not belong to the span of $v_2$ and $v_3$. (Hint: if $v_1 = a v_2 + b v_3$ for some scalars $a$ and $b$, express this as a system of 3 equations on $a$ and $b$ and show that these equations have no simultaneous solution.)

    As the hint suggested, we want to find 2 scalar a and b and express
    \[
    v_1=av_2+cv_3
    \]
   
    \[
\begin{bmatrix}
1 \\
-2 \\
3
\end{bmatrix} =
a .
\begin{bmatrix}
2 \\
0 \\
1
\end{bmatrix} + 
b .
\begin{bmatrix}
3 \\
-2 \\
1
\end{bmatrix}
\]
\[
1 = 2a + 3b
\]
\[
-2 = -2b
\]
\[
3 = a + b
\]
Substituting
\[
b=1
\]
\[
3 = a + 1
\]
Solving we get
\[
a = 2
\]
\[
b=1
\]
But if we put a=2 and b=1 in the first equation we get
\[
1 = 2(2) + 3(1) 
\]
\[
1 \neq 7
\]
This shows that there is no simultaneous solution to these system of equations.
This indicates that $v_1$ does not belong to the span of $v_2$ and $v_3$.
    
    
    
    
    \item[(b)] Similarly show $v_2$ does not belong to the span of $v_1$ and $v_3$, and that $v_3$ does not belong to the span of $v_1$ and $v_2$.

 \begin{enumerate}
     \item[(i)] $v_2$ does not belong to the span of $v_1$ and $v_3$ 
     \[
     \begin{bmatrix}
2 \\
0 \\
1
\end{bmatrix} = a \begin{bmatrix}
1 \\
-2 \\
3
\end{bmatrix} + b . \begin{bmatrix}
3 \\
-2 \\
1
\end{bmatrix}
     \]
     \[
     2 = a + 3b
     \]
      \[
     0 = -2a - 2b
     \]
      \[
     1 = 3a + b
     \]
     Substituting
     \[
     a = -b
     \]
     we get 
     \[
     1 = 3(-b)+b 
     \]
       \[
     b=\frac{-1}{2}
     \]
     Substituting 
     \[
     a = \frac{1}{2}
     \]
     \[
     b = \frac{-1}{2}
     \]
     in first equation we get
     \[
     2 = \frac{1}{2}+3\frac{-1}{2}
     \]
       \[
     2 \neq -1
     \]
     This shows that there is no simultaneous solution to these system of equations.
This indicates that $v_2$ does not belong to the span of $v_1$ and $v_3$.
     
      \item[(i)] $v_3$ does not belong to the span of $v_1$ and $v_2$ 
      \[
      \begin{bmatrix}
3 \\
-2 \\
1
\end{bmatrix} = a \begin{bmatrix}
1 \\
-2 \\
3
\end{bmatrix} + b .
\begin{bmatrix}
2 \\
0 \\
1
\end{bmatrix}
 \]
 \[
 3 = a + 2b
 \]
 \[
 -2 = -2a 
 \]
 \[
 1 = 3a + b
 \]
 Substituting
 \[
 a =1 
 \]
 we get 
 \[
 1 = 3(1) + b 
 \]
  \[
 b =-2
 \]
 Substituting
 \[
 a =1
 \]
 \[
 b=-2
 \]
 in first equation
 \[
 3 = 1 + 2(-2) 
 \]
 \[
 3 \neq -3
 \]
      This shows that there is no simultaneous solution to these system of equations.
This indicates that $v_3$ does not belong to the span of $v_1$ and $v_2$.
 \end{enumerate}

    
    
    \item[(c)] Using (a) and (b), apply Theorem 4.2.5 to conclude that the linear subspace $V = \text{span}(v_1, v_2, v_3)$ in $\mathbb{R}^3$ has dimension equal to 3 (so it coincides with $\mathbb{R}^3$, by Theorem 4.2.8).

    Finding 3 scalar such that \[
    a.v_1 + b.v_2 + c.v_3 = 0
    \]
    \[
    a . \begin{bmatrix}
1 \\
-2 \\
3
\end{bmatrix} + b .
\begin{bmatrix}
2 \\
0 \\
1
\end{bmatrix} + c .
\begin{bmatrix}
3 \\
-2 \\
1
\end{bmatrix} = 0
\]
\[
a + 2b + 3c = 0
\]
\[
-2a - 2c = 0
\]
\[
3a + b + c = 0
\]
Substituting
\[
a = -c
\]
\[
3(-c) +b + c = 0
\]
\[
b = 2c
\]
\[
-c + 2(2c) + 3c = 0
\]
\[
c = 0
\]
\[
b = 0
\]
\[
a = 0
\]
This means $v_1$,  $v_2$,  $v_3$ are linearly independent vectors and they have a dimension of 3
    
\end{enumerate}


